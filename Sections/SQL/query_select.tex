\subsubsection*{SELECT}

We can also retrieve data from our tables using the \textcolor{blue}{\texttt{SELECT}}:

\begin{lstlisting}[style=sql]
    -- Retrieve all columns from bands
    SELECT * FROM bands;

    -- Query Result:
    --     +----+----------------------+
    --     | id | name                 |
    --     +----+----------------------+
    --     | 1  | The Beatles          |
    --     | 2  | The Rolling Stones   |
    --     | 3  | The Who              |
    --     +----+----------------------+
\end{lstlisting}

\noindent
Queries to a table return another table.

\newpage

\noindent
To list a few commands:
\begin{lstlisting}[style=sql]
    -- Retrieve the name column from bands
    SELECT name FROM bands;

    -- Query Result:
    --     +----------------------+
    --     | name                 |
    --     +----------------------+
    --     | The Beatles          |
    --     | The Rolling Stones   |
    --     | The Who              |
    --     +----------------------+
\end{lstlisting}
\begin{lstlisting}[style=sql]
    -- Retrieve the name column from bands, limit to 2
    SELECT name FROM bands LIMIT 1;

    -- Query Result:
    --     +----------------------+
    --     | name                 |
    --     +----------------------+
    --     | The Beatles          |
    --     +----------------------+
\end{lstlisting}
\begin{lstlisting}[style=sql]
    -- Retrieve and give aliases to id and name columns from bands
    SELECT id AS 'ID', name AS 'Band Name'

    -- Query Result:
    --     +----+----------------------+
    --     | ID | Band Name            |
    --     +----+----------------------+
    --     | 1  | The Beatles          |
    --     | 2  | The Rolling Stones   |
    --     | 3  | The Who              |
    --     +----+----------------------+
\end{lstlisting}
\begin{lstlisting}[style=sql]
    -- Order bands by name in descending order
    SELECT * FROM bands ORDER BY name DESC;

    -- Query Result:
    --     +----+----------------------+
    --     | id | name                 |
    --     +----+----------------------+
    --     | 1  | The Who              |
    --     | 2  | The Rolling Stones   |
    --     | 3  | The Beatles          |
    --     +----+----------------------+

    -- Order bands in ascending order
    SELECT * FROM bands ORDER BY name ASC;
    -- which can be shortened to 
    SELECT * FROM bands ORDER BY name; -- as ASC is the default
\end{lstlisting}

\noindent
Here we used the \textcolor{blue}{\texttt{LIMIT}}, \textcolor{blue}{\texttt{AS}}, and \textcolor{blue}{\texttt{ORDER BY (ASC/DESC)}} commands.

\newpage
\noindent
Say we had the table with the following data:

\begin{lstlisting}[style=sql]
    -- DB: school_table
    --
    --     students
    --     +----+----------------------+
    --     | id | name                 |
    --     +----+----------------------+
    --     | 1  | Joe                  |
    --     | 2  | Joe                  |
    --     | 3  | Joe                  |
    --     | 4  | Alvin                |
    --     +----+----------------------+

    -- Retrieve all unique names from students
    USE school_table;
    SELECT DISTINCT name FROM students;

    -- Query Result:
    --     +----------------------+
    --     | name                 |
    --     +----------------------+
    --     | Joe                  |
    --     | Alvin                |
    --     +----------------------+
\end{lstlisting}

\subsection*{UPDATE}
\noindent
To visit our \textcolor{blue}{\texttt{concise\_records}} example again:

\begin{lstlisting}[style=sql]
    -- DB: concise_records
    --
    --     bands
    --     +----+----------------------+
    --     | id | name                 |
    --     +----+----------------------+
    --     | 1  | The Beatles          |
    --     | 2  | The Rolling Stones   |
    --     | 3  | The Who              |
    --     +----+----------------------+
    --    
    --     albums
    --     +----+----------------+--------------+---------+
    --     | id | name           | release_date | band_id |
    --     +----+----------------+--------------+---------+
    --     | 1  | Abbey Road     | 1969         | 1       |
    --     | 2  | Let It Be      | 1970         | 1       |
    --     | 3  | Who's Next     |              | 3       |
    --     +----+----------------+--------------+---------+
\end{lstlisting}

\vspace{1em}
\noindent
We will run the \textcolor{blue}{\texttt{UPDATE}} command to change the \textcolor{blue}{\texttt{release\_date}}.

\newpage

\noindent
To Change the release date of \textcolor{blue}{\texttt{Who's Next} }to 1971:
\begin{lstlisting}[style=sql]
    -- We could do
    UPDATE albums
    SET release_date = '1971'

    -- But that would result in all albums having the same release date
    -- instead we use WHERE
    UPDATE albums SET release_date = '1971' WHERE name = 'Who''s Next';

    -- DB: concise_records
    --
    --     albums
    --     +----+----------------+--------------+---------+
    --     | id | name           | release_date | band_id |
    --     +----+----------------+--------------+---------+
    --     | 1  | Abbey Road     | 1969         | 1       |
    --     | 2  | Let It Be      | 1970         | 1       |
    --     | 3  | Who's Next     | 1971         | 3       |
    --     +----+----------------+--------------+---------+
\end{lstlisting}


