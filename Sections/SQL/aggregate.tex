
\subsection{Aggregate Functions}

\noindent
We can aggregate, i.e., decipher trends in our data via computational functions. Our data:

\begin{lstlisting}[style=sql]
    -- DB: concise_records
    --
    --     bands
    --     +----+----------------------+
    --     | id | name                 |
    --     +----+----------------------+
    --     | 1  | The Beatles          |
    --     | 2  | The Rolling Stones   |
    --     | 3  | The Who              |
    --     +----+----------------------+
    --    
    --     albums
    --     +----+----------------+--------------+---------+
    --     | id | name           | release_date | band_id |
    --     +----+----------------+--------------+---------+
    --     | 1  | Abbey Road     | 1969         | 1       |
    --     | 2  | Let It Be      | 1970         | 1       |
    --     | 3  | Who's Next     | 1971         | 3       |
    --     +----+----------------+--------------+---------+
\end{lstlisting}

\noindent
\textbf{Examples:}\\

\noindent
1. \textcolor{blue}{\texttt{AVG()}}
\begin{lstlisting}[style=sql]
    -- Retrieve the average release date of albums
    SELECT AVG(release_date) FROM albums;

    -- Query Result:
    --     +------------------+
    --     | AVG(release_date)|
    --     +------------------+
    --     | 1970             |
    --     +------------------+
\end{lstlisting}
\noindent
2. \textcolor{blue}{\texttt{SUM()}}
\begin{lstlisting}[style=sql]
    -- Retrieve the sum of release dates of albums
    SELECT SUM(release_date) FROM albums;

    -- Query Result:
    --     +------------------+
    --     | SUM(release_date)|
    --     +------------------+
    --     | 5910             |
    --     +------------------+
\end{lstlisting}
\noindent
3. \textcolor{blue}{\texttt{COUNT()}}
\begin{lstlisting}[style=sql]
    -- Retrieve the count of albums
    SELECT COUNT(*) FROM albums;

    -- Query Result:
    --     +----------+
    --     | COUNT(*) |
    --     +----------+
    --     | 3        |
    --     +----------+
\end{lstlisting}

\noindent
There are plenty of aggregate functions, which are provided by the \underline{\textbf{MySQL documentation}.}\\

\vspace{1em}

\begin{tabular}{|c|p{8cm}|}
    \hline
    \rowcolor{OliveGreen!10} \textbf{NAME} & \textbf{DESCRIPTION}                             \\
    \hline
    AVG()                                  & Return the average value of the argument         \\
    \rowcolor{black!5} BIT\_AND()          & Return bitwise AND                               \\
    BIT\_OR()                              & Return bitwise OR                                \\
    \rowcolor{black!5} BIT\_XOR()          & Return bitwise XOR                               \\
    COUNT()                                & Return a count of the number of rows returned    \\
    \rowcolor{black!5} COUNT(DISTINCT)     & Return the count of a number of different values \\
    GROUP\_CONCAT()                        & Return a concatenated string                     \\
    \rowcolor{black!5} JSON\_ARRAYAGG()    & Return result set as a single JSON array         \\
    JSON\_OBJECTAGG()                      & Return result set as a single JSON object        \\
    \rowcolor{black!5} MAX()               & Return the maximum value                         \\
    MIN()                                  & Return the minimum value                         \\
    \rowcolor{black!5} STD()               & Return the population standard deviation         \\
    STDDEV()                               & Return the population standard deviation         \\
    \rowcolor{black!5} STDDEV\_POP()       & Return the population standard deviation         \\
    STDDEV\_SAMP()                         & Return the sample standard deviation             \\
    \rowcolor{black!5} SUM()               & Return the sum                                   \\
    VAR\_POP()                             & Return the population standard variance          \\
    \rowcolor{black!5} VAR\_SAMP()         & Return the sample variance                       \\
    VARIANCE()                             & Return the population standard variance          \\
    \hline
\end{tabular}

\vspace{1em}
\begin{centering}
    \textit{This table is taken from the MySQL documentation.\\}
\end{centering}

\noindent
