\subsection{Mutating Data \& Filtering}
\noindent
Again, visiting our \textcolor{blue}{\texttt{concise\_records}} example:

\begin{lstlisting}[style=sql]
    -- DB: concise_records
    --
    --     bands
    --     +----+----------------------+
    --     | id | name                 |
    --     +----+----------------------+
    --     | 1  | The Beatles          |
    --     | 2  | The Rolling Stones   |
    --     | 3  | The Who              |
    --     +----+----------------------+
    --    
    --     albums
    --     +----+----------------+--------------+---------+
    --     | id | name           | release_date | band_id |
    --     +----+----------------+--------------+---------+
    --     | 1  | Abbey Road     | 1969         | 1       |
    --     | 2  | Let It Be      | 1970         | 1       |
    --     | 3  | Who's Next     |              | 3       |
    --     +----+----------------+--------------+---------+
\end{lstlisting}

\vspace{1em}
\noindent
We will begin to update and filter for specific entries.

\newpage

\noindent
To change the release date of \textcolor{blue}{\texttt{Who's Next} }to `1971':
\begin{lstlisting}[style=sql]
    UPDATE albums SET release_date = '1971';
    -- But that would result in all albums having the same release date
    -- instead we use WHERE

    UPDATE albums SET release_date = '1971' WHERE id = 3;

    -- DB: concise_records
    --
    --     albums
    --     +----+----------------+--------------+---------+
    --     | id | name           | release_date | band_id |
    --     +----+----------------+--------------+---------+
    --     | 1  | Abbey Road     | 1969         | 1       |
    --     | 2  | Let It Be      | 1970         | 1       |
    --     | 3  | Who's Next     | 1971         | 3       |
    --     +----+----------------+--------------+---------+
\end{lstlisting}

\noindent
We used \textcolor{blue}{\texttt{id}}, as the \textcolor{blue}{\texttt{name}} could change or
be duplicated. \textcolor{blue}{\texttt{WHERE}} filters rows based on conditions.\\
\textbf{Examples:}\\

\noindent
1. \textcolor{blue}{\texttt{>, <, >=, <=}}
\begin{lstlisting}[style=sql]
    -- Select all albums with a release date greater than 1969
    SELECT * FROM albums WHERE release_date > 1969;

    -- Query Result:
    --     +----+----------------+--------------+---------+
    --     | id | name           | release_date | band_id |
    --     +----+----------------+--------------+---------+
    --     | 2  | Let It Be      | 1970         | 1       |
    --     | 3  | Who's Next     | 1971         | 3       |
    --     +----+----------------+--------------+---------+
    
\end{lstlisting}

\noindent
2. \textcolor{blue}{\texttt{LIKE}}
\begin{lstlisting}[style=sql]
    -- Select all albums that contain 'be' in their name
    SELECT * FROM albums WHERE name LIKE '%be%';

    -- Query Result:
    --     +----+----------------+--------------+---------+
    --     | id | name           | release_date | band_id |
    --     +----+----------------+--------------+---------+
    --     | 2  | Let It Be      | 1970         | 1       |
    --     +----+----------------+--------------+---------+
    
\end{lstlisting}
\noindent
3. \textcolor{blue}{\texttt{OR}}
\begin{lstlisting}[style=sql]
    -- Albums that contain 'x' in their name OR have a release date of 1969 
    SELECT * FROM albums WHERE name LIKE '%x%' OR band_id = 1;

    -- Query Result:
    --     +----+----------------+--------------+---------+
    --     | id | name           | release_date | band_id |
    --     +----+----------------+--------------+---------+
    --     | 1  | Abbey Road     | 1969         | 1       |
    --     | 3  | Who's Next     | 1971         | 3       |
    --     +----+----------------+--------------+---------+
\end{lstlisting}

\newpage

\noindent
4. \textcolor{blue}{\texttt{AND}}
\begin{lstlisting}[style=sql]
    -- Select all albums that released in 1969 and have an band_id of 1
    SELECT * FROM albums WHERE release_date = 1969 AND band_id = 1;

    -- Query Result:
    --     +----+----------------+--------------+---------+
    --     | id | name           | release_date | band_id |
    --     +----+----------------+--------------+---------+
    --     | 1  | Abbey Road     | 1969         | 1       |
    --     +----+----------------+--------------+---------+
\end{lstlisting}

\noindent
5. \textcolor{blue}{\texttt{BETWEEN}}
\begin{lstlisting}[style=sql]
    -- Select all albums that released between 1969 and 1971 (incluvise)
    SELECT * FROM albums WHERE release_date BETWEEN 1900 AND 1980;

    -- Query Result:
    --     +----+----------------+--------------+---------+
    --     | id | name           | release_date | band_id |
    --     +----+----------------+--------------+---------+
    --     | 1  | Abbey Road     | 1969         | 1       |
    --     | 2  | Let It Be      | 1970         | 1       |
    --     | 3  | Who's Next     | 1971         | 3       |
    --     +----+----------------+--------------+---------+

\end{lstlisting}

\noindent
6. \textcolor{blue}{\texttt{IS NULL}}

\begin{lstlisting}[style=sql]
    -- Add a new album with a NULL release date
    INSERT INTO albums (name, band_id) VALUES ('The Wall', 2);

    -- albums
    -- +----+----------------+--------------+---------+
    -- | id | name           | release_date | band_id |
    -- +----+----------------+--------------+---------+
    -- | 1  | Abbey Road     | 1969         | 1       |
    -- | 2  | Let It Be      | 1970         | 1       |
    -- | 3  | Who's Next     | 1971         | 3       |
    -- | 4  | The Wall       |              | 2       |
    -- +----+----------------+--------------+---------+

    -- Select all albums with a NULL release date
    SELECT * FROM albums WHERE release_date IS NULL;

    -- Query Result:
    -- +----+------------+--------------+---------+
    -- | id | name       | release_date | band_id |
    -- +----+------------+--------------+---------+
    -- | 4  | The Wall   |              | 2       |
    -- +----+------------+--------------+---------+
    
\end{lstlisting}

\newpage

\noindent
7. \textcolor{blue}{\texttt{DELETE}}

\begin{lstlisting}[style=sql]
    -- Delete all albums
    DELETE FROM albums;
    -- lets not do this though, instead

    -- Delete all albums with name 'The Wall' (added in previous example)
    DELETE FROM albums WHERE name = 'The Wall';

    -- albums
    -- +----+----------------+--------------+---------+
    -- | id | name           | release_date | band_id |
    -- +----+----------------+--------------+---------+
    -- | 1  | Abbey Road     | 1969         | 1       |
    -- | 2  | Let It Be      | 1970         | 1       |
    -- | 3  | Who's Next     | 1971         | 3       |
    -- +----+----------------+--------------+---------+

\end{lstlisting}