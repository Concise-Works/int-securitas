\subsection{Creating Database \& Tables}





SQL stands for ``\textbf{\textit{Structured Query Language}},'' used to query against databases with tables
containing columns of data, which most often relate to each other.\\

\noindent
Using key words like \textcolor{blue}{\texttt{SELECT}}, \textcolor{blue}{\texttt{FROM}}, \textcolor{blue}{\texttt{WHERE}},
ignoring case. It's good practice to use \textbf{all caps for SQL keywords}, and \textbf{lowercase for table and column names}.
Here's a simple example:\\

\begin{lstlisting}[style=sql]
    SELECT * FROM my_table
\end{lstlisting}
\begin{centering}
    \textit{Selects all (*) columns from the table \textcolor{blue}{\texttt{my\_table}}.}\\
\end{centering}
\vspace{1em}
\noindent
Now, we are a record company with bands, albums, and songs:

\begin{lstlisting}[style=sql]
    CREATE DATABASE test; -- creating a test database
    DROP DATABASE test; -- deleting the test database


    CREATE DATABASE concise_records; -- creating our database 
    USE concise_records; -- selecting our database to run commands on it


    CREATE TABLE bandds (); -- creating a table for our bands
    DROP TABLE bandds; -- deleting it because of our typo
    

    -- Create bands table: artist name (at most 255 characters), CANNOT be NULL/EMPTY
    CREATE TABLE bands (
        name VARCHAR(255) NOT NULL
    );


    -- Add id column to bands, auto increment, not NULL, make this column important
    ALTER TABLE bands 
    ADD COLUMN id INT NOT NULL AUTO_INCREMENT PRIMARY KEY;
\end{lstlisting}

\vspace{1em}
\noindent
We created the database, \textcolor{blue}{\texttt{concise\_records}}, and a table \textcolor{blue}{\texttt{bands}}
with two columns: \textcolor{blue}{\texttt{name}} and \textcolor{blue}{\texttt{id}}.\\


\newpage

\noindent
The \textcolor{blue}{\texttt{PRIMARY KEY}} acts as an ID for each row, useful for
drawing a thread of relationships between tables where the ID is present.\\

\begin{Def}[Primary Key]
    .A column which identifies each row in a table. It must be unique, and it cannot be \textcolor{blue}{\texttt{NULL}}.
\end{Def}

\noindent
To create our \textcolor{blue}{albums} table:

\begin{lstlisting}[style=sql]
    -- Create albums table: album id, album names, release dates (optional)
    CREATE TABLE albums (
        id INT NOT NULL AUTO_INCREMENT,
        name VARCHAR(255) NOT NULL,
        release_date DATE,
        PRIMARY KEY (id)
    );


    -- We need a way to link bands to albums
    -- Create a column in albums pointing to bands' id column
    ALTER TABLE albums
    ADD COLUMN band_id INT NOT NULL FOREIGN KEY REFERENCES bands(id);

\end{lstlisting}

\vspace{1em}
\noindent
We created the \textcolor{blue}{albums} table with three columns: \textcolor{blue}{\texttt{id}}, \textcolor{blue}{\texttt{name}}, and \textcolor{blue}{\texttt{release\_date}},
and \textcolor{blue}{\texttt{band\_id}}. A \textcolor{blue}{\texttt{FOREIGN KEY}} uses the \textcolor{blue}{\texttt{PRIMARY KEY}} of another table to establish a relationship between them.\\

\begin{Def}[Foreign Key]
    .A column that references another table's \textcolor{blue}{\texttt{PRIMARY KEY}}.
\end{Def}

\noindent
So far we have:
\begin{lstlisting}[style=sql]
    -- DB: concise_records
    --
    --     bands
    --     +----+----------------+
    --     | id | name           |
    --     +----+----------------+
    --     |    |                |
    --     +----+----------------+
    --
    --     albums
    --     +----+----------------+--------------+---------+
    --     | id | name           | release_date | band_id |
    --     +----+----------------+--------------+---------+
    --     |    |                |              |         |
    --     +----+----------------+--------------+---------+
\end{lstlisting}

\noindent
The database \textcolor{blue}{\texttt{concise\_records}} with our tables \textcolor{blue}{\texttt{bands}} and \textcolor{blue}{\texttt{albums}}.\\

\newpage

\noindent
Let's begin to add data to our tables:

\begin{lstlisting}[style=sql]
    -- Insert 'The Beatles', 'The Rolling Stones', 'The Who' into bands
    INSERT INTO bands (name) VALUES ('The Beatles');
    INSERT INTO bands (name) VALUES ('The Rolling Stones'), ('The Who');

    -- DB: concise_records
    --
    --     bands
    --     +----+----------------------+
    --     | id | name                 |
    --     +----+----------------------+
    --     | 1  | The Beatles          |
    --     | 2  | The Rolling Stones   |
    --     | 3  | The Who              |
    --     +----+----------------------+

    -- Insert 'Abbey Road', 'Let It Bleed', 'Who's Next' into albums

    INSERT INTO albums (name, release_date, band_id) VALUES ('Abbey Road', '1969', 1);
    INSERT INTO albums (name, release_date, band_id) VALUES ('Let Be', '1970', 1);
    INSERT INTO albums (name, band_id) VALUES ('Who''s Next', 3);

    -- DB: concise_records
    --
    --     albums
    --     +----+----------------+--------------+---------+
    --     | id | name           | release_date | band_id |
    --     +----+----------------+--------------+---------+
    --     | 1  | Abbey Road     | 1969         | 1       |
    --     | 2  | Let It Be      | 1970         | 1       |
    --     | 3  | Who's Next     |              | 3       |
    --     +----+----------------+--------------+---------+

\end{lstlisting}


\noindent
\underline{Single quotes denote \textcolor{blue}{strings}.} Double single quotes in strings act as single quotes, seen in
the above with \textcolor{blue}{\texttt{'Who''s next'}}.\\







