\subsection{SQL Basics}





SQL stands for ``\textbf{\textit{Structured Query Language}},'' used to query against databases with tables
containing columns of data, which most often relate to each other.\\

\noindent
Using select words like \textcolor{blue}{\texttt{SELECT}}, \textcolor{blue}{\texttt{FROM}}, \textcolor{blue}{\texttt{WHERE}},
ignoring case. It's good practice to use \textbf{all caps for SQL keywords}, and \textbf{lowercase for table and column names}.
Here's a simple example:\\

\begin{lstlisting}[style=sql]
    SELECT * FROM my_table
\end{lstlisting}
\begin{centering}
    \textit{Selects all (*) columns from the table \textcolor{blue}{\texttt{my\_table}}.}\\
\end{centering}

\vspace{1em}
\noindent
Now, we are a record company with bands, albums, and songs:\\


\begin{lstlisting}[style=sql]
    CREATE DATABASE test; -- creating a test database
    DROP DATABASE test; -- deleting the test database

    CREATE DATABASE concise_records; -- creating our database 
    USE concise_records; -- selecting our database to run commands on it

    CREATE TABLE bandds (); -- creating a table for our bands
    DROP TABLE bandds; -- deleting it because of our typo
    
    -- Create bands table: artist names (at most 255 characters), CANNOT be NULL/EMPTY
    CREATE TABLE bands (
        name VARCHAR(255) NOT NULL
    );

    -- Add id column to bands, auto increment, not NULL, make this column important
    ALTER TABLE bands 
    ADD COLUMN id INT NOT NULL AUTO_INCREMENT PRIMARY KEY;
\end{lstlisting}

\vspace{1em}
\noindent
We created the database, \textcolor{blue}{\texttt{concise\_records}}, and a table \textcolor{blue}{\texttt{bands}}
with two columns: \textcolor{blue}{\texttt{name}} and \textcolor{blue}{\texttt{id}}.\\


\newpage

\noindent
The \textcolor{blue}{\texttt{PRIMARY KEY}} acts as an ID for each row, useful for
drawing a thread of relationships between tables where the ID is present.\\

\begin{Def}[Primary Key]
    .A column which identifies each row in a table. It must be unique, and it cannot be NULL.
\end{Def}

\noindent
To create our albums table:

\begin{lstlisting}[style=sql]
    CREATE TABLE albums (
        id INT NOT NULL AUTO_INCREMENT PRIMARY KEY,
        name VARCHAR(255) NOT NULL,
        release_date DATE NOT NULL,
    );

\end{lstlisting}



