\subsection{Queries}
To retrieve data, we use \textcolor{blue}{\texttt{SELECT}}:

\begin{lstlisting}[style=sql]
    -- Retrieve all columns from bands
    SELECT * FROM bands;

    -- Query Result:
    --     +----+----------------------+
    --     | id | name                 |
    --     +----+----------------------+
    --     | 1  | The Beatles          |
    --     | 2  | The Rolling Stones   |
    --     | 3  | The Who              |
    --     +----+----------------------+
\end{lstlisting}

\noindent
Queries to a table return another table.

\newpage

\noindent
\textbf{Examples:}\\

\noindent
1. \textcolor{blue}{\texttt{SELECT} \textit{column}}
\begin{lstlisting}[style=sql]
    -- Retrieve the name column from bands
    SELECT name FROM bands;

    -- Query Result:
    --     +----------------------+
    --     | name                 |
    --     +----------------------+
    --     | The Beatles          |
    --     | The Rolling Stones   |
    --     | The Who              |
    --     +----------------------+
\end{lstlisting}
\noindent
2. \textcolor{blue}{\texttt{LIMIT}}
\begin{lstlisting}[style=sql]
    -- Retrieve the name column from bands, limit to 1    
    SELECT name FROM bands LIMIT 1;

    -- Query Result:
    --     +----------------------+
    --     | name                 |
    --     +----------------------+
    --     | The Beatles          |
    --     +----------------------+
\end{lstlisting}
\noindent
3. \textcolor{blue}{\texttt{AS}}
\begin{lstlisting}[style=sql]
    -- Retrieve and give aliases to id and name columns from bands, limit to 1
    SELECT id AS 'ID', name AS 'Band Name' LIMIT 1;

    -- Query Result:
    --     +----+----------------------+
    --     | ID | Band Name            |
    --     +----+----------------------+
    --     | 1  | The Beatles          |
    --     +----+----------------------+
\end{lstlisting}
\noindent
4. \textcolor{blue}{\texttt{ORDER BY (DESC/ASC)}}
\begin{lstlisting}[style=sql]
    -- Order bands by name in descending order
    SELECT * FROM bands ORDER BY name DESC;

    -- Query Result:
    --     +----+----------------------+
    --     | id | name                 |
    --     +----+----------------------+
    --     | 1  | The Who              |
    --     | 2  | The Rolling Stones   |
    --     | 3  | The Beatles          |
    --     +----+----------------------+

    -- Order bands in ascending order
    SELECT * FROM bands ORDER BY name ASC;
    -- which can be shortened to 
    SELECT * FROM bands ORDER BY name; -- as ASC is the default
\end{lstlisting}

\noindent
Here we used the \textcolor{blue}{\texttt{LIMIT}}, \textcolor{blue}{\texttt{AS}}, and \textcolor{blue}{\texttt{ORDER BY (ASC/DESC)}} commands.

\newpage
\noindent
5. \textcolor{blue}{\texttt{DISTINCT}}

\begin{lstlisting}[style=sql]
    -- Say we had the following table:
    -- DB: school_table
    --
    --     students
    --     +----+----------------------+
    --     | id | name                 |
    --     +----+----------------------+
    --     | 1  | Joe                  |
    --     | 2  | Joe                  |
    --     | 3  | Joe                  |
    --     | 4  | Alvin                |
    --     +----+----------------------+

    -- Retrieve all unique names from students
    USE school_table;
    SELECT DISTINCT name FROM students;

    -- Query Result:
    --     +----------------------+
    --     | name                 |
    --     +----------------------+
    --     | Joe                  |
    --     | Alvin                |
    --     +----------------------+
\end{lstlisting}